\documentclass[12pt,a4paper]{article}
\usepackage[utf8]{inputenc}
\usepackage[T1]{fontenc}
\usepackage[provide=*,french]{babel}
\usepackage{graphicx}
\usepackage{geometry}
\usepackage{hyperref}
\usepackage[all]{hypcap}

\geometry{margin=2.5cm}

\begin{document}

% Page de garde
\begin{titlepage}
    \begin{center}
        \vspace*{2cm}
        {\huge\bfseries Devoir 3\par}
        {INFO4305\par}
        \vspace{2cm}
        {\Large Alec Jones\par}
        {\large A00216262\par}
        \vfill
    \end{center}
\end{titlepage}

\tableofcontents
\newpage

% Introduction
\section{Introduction}

% Objectif
\section{Objectif du TP}
 [Les objectifs du TP]

% Corps du rapport
\newpage
\section{Déroulement du TP}
\subsection{Partie 1}
Premièrement on doit d'abord installer Gpg4win, j'ai installer le logiciel à
l'aide du manager de paquets WinGet (voir la figure \ref{gpg4win}).

\begin{figure}[h]
    \centering
    \includegraphics[width=0.8\textwidth]{../img/gpg4win.png}
    \caption{Installation de Gpg4win}
    \label{gpg4win}
\end{figure}

\subsection{Partie 2}
\subsubsection{Partie a}
Pour créer une paire de clé à l'aide de l'interface graphique, on doit ouvirir le logiciel
Kleopatra, ensuite on selectionne l'option new key pair (voir la figure \ref{kleopatra}).

\begin{figure}[h]
    \centering
    \includegraphics[width=0.8\textwidth]{../img/kleopatra.png}
    \caption{Menu initial Kleopatra}
    \label{kleopatra}
\end{figure}

Ensuite, on doit sélectionné options avancés et puis rsa2048.
On doit aussi remplir notre nom et notre courriel pour le certificat (voir \ref{newKey}).

\begin{figure}[ht]
    \centering
    \includegraphics[width=0.8\textwidth]{../img/newKey.png}
    \caption{Création de clés dans Kleopatra}
    \label{newKey}
\end{figure}

\subsubsection{Partie b}
Pour créer une paire de clé en ligne de commande, vous pouvez utiliser la commande suivante dans un terminal :

\begin{verbatim}
gpg --full-generate-key
\end{verbatim}

Cette commande lance un assistant interactif vous permettant de choisir le type de clé, la longueur (par exemple, rsa2048 ou rsa4096)
et de renseigner les informations nécessaires (nom, adresse électronique, etc.). Une fois terminé, votre paire de clés sera générée et stockée dans votre trousseau GPG (voir figure \ref{newKey_cli}).

\begin{figure}[ht]
    \centering
    \includegraphics[width=0.8\textwidth]{../img/newKey_cli.png}
    \caption{Création de clés à l'aide de la ligne de commande}
    \label{newKey_cli}
\end{figure}

\subsection{Partie 3}
Pour Lister notre trousseau de clés, on peut utiliser la commande suivante:
\begin{verbatim}
    gpg --list-keys
\end{verbatim}
En exécutant la commande on appercois les deux clés créers précédément (voir figure \ref{listKeys}).

\begin{figure}[ht]
    \centering
    \includegraphics[width=0.8\textwidth]{../img/listKeys.png}
    \caption{Liste des clés}
    \label{listKeys}
\end{figure}

\subsection{Partie 4}
Pour exporter les clés publiques, on utilise la commande:
\begin{verbatim}
    gpg --armor --output maclé.asc --export UserID
\end{verbatim}

% Conclusion
\section{Observation, interprétation et conclusion}
 [Vos observations et conclusions]
\begin{itemize}
    \item Objectifs atteints ou non
    \item Ce que vous avez accompli
    \item Ce que vous avez compris
\end{itemize}

\end{document}
