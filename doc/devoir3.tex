\documentclass[12pt,a4paper]{article}
\usepackage[utf8]{inputenc}
\usepackage[T1]{fontenc}
\usepackage[provide=*,french]{babel}
\usepackage{graphicx}
\usepackage{geometry}
\usepackage{hyperref}
\usepackage[all]{hypcap}
\usepackage{enumitem}
\usepackage{fancyvrb}

\geometry{margin=2.5cm}

\begin{document}

% Page de garde
\begin{titlepage}
    \begin{center}
        \vspace*{2cm}
        {\huge\bfseries Devoir 3\par}
        {INFO4305\par}
        \vspace{2cm}
        {\Large Alec Jones\par}
        {\large A00216262\par}
        \vfill
    \end{center}
\end{titlepage}

\tableofcontents
\newpage

% Introduction
\section{Introduction}

% Objectif
\section{Objectif du TP}
 [Les objectifs du TP]

% Corps du rapport
\newpage
\section{Déroulement du TP}
\subsection{Partie 1}
Premièrement on doit d'abord installer Gpg4win, j'ai installer le logiciel à
l'aide du manager de paquets WinGet (voir la figure \ref{gpg4win}).

\begin{figure}[h]
    \centering
    \includegraphics[width=0.8\textwidth]{../img/gpg4win.png}
    \caption{Installation de Gpg4win}
    \label{gpg4win}
\end{figure}

\subsection{Partie 2}
\subsubsection{Partie a}
Pour créer une paire de clé à l'aide de l'interface graphique, on doit ouvirir le logiciel
Kleopatra, ensuite on selectionne l'option new key pair (voir la figure \ref{kleopatra}).

\begin{figure}[h]
    \centering
    \includegraphics[width=0.8\textwidth]{../img/kleopatra.png}
    \caption{Menu initial Kleopatra}
    \label{kleopatra}
\end{figure}

Ensuite, on doit sélectionné options avancés et puis rsa2048.
On doit aussi remplir notre nom et notre courriel pour le certificat (voir \ref{newKey}).

\begin{figure}[ht]
    \centering
    \includegraphics[width=0.8\textwidth]{../img/newKey.png}
    \caption{Création de clés dans Kleopatra}
    \label{newKey}
\end{figure}

\subsubsection{Partie b}
Pour créer une paire de clé en ligne de commande, vous pouvez utiliser la commande suivante dans un terminal :

\begin{verbatim}
gpg --full-generate-key
\end{verbatim}

Cette commande lance un assistant interactif vous permettant de choisir le type de clé, la longueur (par exemple, rsa2048 ou rsa4096)
et de renseigner les informations nécessaires (nom, adresse électronique, etc.). Une fois terminé, votre paire de clés sera générée et stockée dans votre trousseau GPG (voir figure \ref{newKey_cli}).

\begin{figure}[ht]
    \centering
    \includegraphics[width=0.8\textwidth]{../img/newKey_cli.png}
    \caption{Création de clés à l'aide de la ligne de commande}
    \label{newKey_cli}
\end{figure}

\subsection{Partie 3}
Pour Lister notre trousseau de clés, on peut utiliser la commande suivante:
\begin{verbatim}
    gpg --list-keys
\end{verbatim}
En exécutant la commande on appercois les deux clés créers précédément (voir figure \ref{listKeys}).

\begin{figure}[ht]
    \centering
    \includegraphics[width=0.8\textwidth]{../img/listKeys.png}
    \caption{Liste des clés}
    \label{listKeys}
\end{figure}

\subsection{Partie 4}
Pour exporter les clés publiques, on utilise la commande:
\begin{verbatim}
    gpg --armor --output maclé.asc --export UserID
\end{verbatim}
Puisqu'on spécifie le UserID, par example eaj0814@umoncton.ca,
qui à été utiliser dans les deux clés, on obtient la concaténation des deux dans un fichier.
On remarque alors que maclé.asc est éffectivement la concaténation des deux clés publiques créer tout à l'heure.

\subsection{Partie 5}
Pour chiffrer un fichier, on utilise la commande suivante (voir figure \ref{chiffre} pour un example):
\begin{verbatim}
    gpg -er UserID document.txt
\end{verbatim}

\begin{figure}[ht]
    \centering
    \includegraphics[width=0.8\textwidth]{../img/chiffre.png}
    \caption{text clair à gauche, chiffrer à droite}
    \label{chiffre}
\end{figure}

\subsection{Partie 6}
Si on voulais ensuite déchiffrer ce texte, on utiliserais la commande suivant :
\begin{verbatim}
    gpg --output doc --decrypt doc.gpg
\end{verbatim}

\subsection{Partie 7}
Pour signer un document et laisser le text en clair, on utilise la commande suivante (Voir la figure \ref{sign} pour example de résultat):
\begin{verbatim}
    gpg --clearsign document.txt
\end{verbatim}

\begin{figure}[ht]
    \centering
    \includegraphics[width=0.8\textwidth]{../img/sign.png}
    \caption{Document signé}
    \label{sign}
\end{figure}

\subsection{Partie 8}
Pour vérifier la signature, on utilise la commande suivante (voir figure \ref{verif}):
\begin{verbatim}
    gpg --verify document.txt.asc
\end{verbatim}

\begin{figure}[ht]
    \centering
    \includegraphics[width=0.8\textwidth]{../img/verif.png}
    \caption{Vérification de la signature}
    \label{verif}
\end{figure}

\subsection{Partie 9}
\subsubsection{Partie a}
L'utilité de la signature dans ce contexte est de garantir l'intégrité du document et d'assurer que le document n'a pas été modifié depuis sa signature.
En vérifiant la signature, on peut s'assurer que le document provient bien de la personne qui l'a signé et qu'il n'a pas été altéré.
(voir figure \ref{verif_modif} pour un example de document modifier).

\begin{figure}[ht]
    \centering
    \includegraphics[width=0.8\textwidth]{../img/verif_modif.png}
    \caption{Vérification de la signature sur un document modifié, le premier mot a été enlevé}
    \label{verif_modif}
\end{figure}

\subsubsection{Partie b - Communication avec un camarade de classe}
\begin{enumerate}[label=\Roman*]
    \item Exporter votre clé publique. L'envoyer à un camarade de classe. J'ai réutiliser la commande suivante:
          \begin{verbatim}
        gpg --armor --output maclé.asc --export UserID
    \end{verbatim}
          J'ai ensuite envoyer le fichier maclé.asc à mon camarade de classe par courriel.

    \item Récupérer la clé publique de votre camarade et l'importer dans votre base de clés.
          Pour se faire j'ai utiliser la commande suivante (voir figure \ref{import}):
          \begin{verbatim}
        gpg --import maclé.asc
    \end{verbatim}

          \begin{figure}[ht]
              \centering
              \includegraphics[width=0.8\textwidth]{../img/importCle.png}
              \caption{Importation de la clé publique de mon camarade}
              \label{import}
          \end{figure}


    \item Visualiser votre base de clés.
          Pour se faire j'ai utiliser la commande suivante:
          \begin{verbatim}
        gpg --list-keys
    \end{verbatim}
          On remarque que la clé de mon camarade a bien été importer (voir figure \ref{listKeysSimon}).

          \begin{figure}[ht]
              \centering
              \includegraphics[width=0.8\textwidth]{../img/listKeysSimon.png}
              \caption{Liste des clés après importation de la clé publique de mon camarade}
              \label{listKeysSimon}
          \end{figure}

    \item Chiffrer un message à destination d'un camarade et lui envoyer.
          Pour se faire j'ai utiliser la commande suivante:
          \begin{verbatim}
            gpg -er UserID paragraphe.txt
          \end{verbatim}
          J'ai ensuite envoyer le fichier paragraphe.txt.gpg à mon camarade de classe par discord.

    \item Déchiffrer un message reçu d'un camarade.
          Pour se faire j'ai utiliser la commande suivante (Voir figure \ref{dechiffrer}):
          \begin{verbatim}
            gpg --decrypt doc.gpg
            \end{verbatim}

          \begin{figure}[ht]
              \centering
              \includegraphics[width=0.8\textwidth]{../img/dechiffrer.png}
              \caption{Déchiffrement du message de mon camarade}
              \label{dechiffrer}
          \end{figure}

    \item Calculer la signature électronique d'un message et l'expédier à un camarade.
          Pour se faire j'ai utiliser la commande suivante:
          \begin{verbatim}
                gpg --clearsign paragraphe.txt
            \end{verbatim}
          J'ai ensuite envoyer le fichier paragraphe.txt.asc à mon camarade de classe par discord.

    \item Récupérer un message signé d'un camarade et vérifier sa provenance/intégrité.
          Après avoir reçu le message signé de mon camarade, j'ai utiliser la commande suivante pour vérifier la signature:
          \begin{verbatim}
                gpg --verify paragraphe.txt.asc
            \end{verbatim}
          On remarque que le message a bien été signé par mon camarade (voir figure \ref{verifSimon}).

          \begin{figure}[ht]
              \centering
              \includegraphics[width=0.8\textwidth]{../img/verifSimon.png}
              \caption{Vérification de la signature du message de mon camarade}
              \label{verifSimon}
          \end{figure}
\end{enumerate}

\subsection{Partie 10}
Pour créer un certificat à l'aide de openssl, on utilisera la commande suivante:
\begin{Verbatim}[fontsize=\footnotesize]
    openssl req -newkey rsa:2048 -keyout domain.key -out domain.csr
\end{Verbatim}

Disecton cette commande (voir figure \ref{opensslCert} pour la sortie):
\begin{itemize}
    \item \texttt{openssl req} : Indique que nous voulons créer une demande de certificat.
    \item \texttt{-newkey rsa:2048} : Crée une nouvelle clé RSA de 2048 bits.
    \item \texttt{-keyout domain.key} : Spécifie le fichier de sortie pour la clé privée.
    \item \texttt{-out domain.csr} : Spécifie le fichier de sortie pour la demande de signature de certificat (CSR).
    \item \texttt{-days 365} : Définit la durée de validité du certificat à 365 jours.
    \item \texttt{-sha256} : Utilise l'algorithme de hachage SHA-256 pour le certificat.
\end{itemize}

\begin{figure}[ht]
    \centering
    \includegraphics[width=0.8\textwidth]{../img/opensslCert.png}
    \caption{Création d'un certificat avec OpenSSL}
    \label{opensslCert}
\end{figure}

Cette commande génère deux fichiers :
\begin{itemize}
    \item \texttt{domain.key} : La clé privée.
    \item \texttt{domain.csr} : La demande de signature de certificat (CSR).
\end{itemize}

Normalement, ce serait à une autorité de certification (CA) de signer le certificat,
mais pour les besoins de ce TP, nous allons le signer nous-mêmes.

\subsection{Partie 11}
Pour signer le certificat avec la clé privée, on utilise la commande suivante:
\begin{Verbatim}[fontsize=\footnotesize]
    openssl x509 -signkey domain.key -in domain.csr -req -days 365 -out domain.crt
\end{Verbatim}

Disecton cette commande (voir figure \ref{opensslCertSign} pour la sortie):
\begin{itemize}
    \item \texttt{openssl x509} : Indique que nous voulons travailler avec des certificats X.509.
    \item \texttt{-signkey domain.key} : Spécifie la clé privée à utiliser pour signer le certificat.
    \item \texttt{-in domain.csr} : Spécifie le fichier d'entrée (CSR) à signer.
    \item \texttt{-req} : Indique que nous travaillons avec une demande de signature de certificat.
    \item \texttt{-days 365} : Définit la durée de validité du certificat à 365 jours.
    \item \texttt{-out domain.crt} : Spécifie le fichier de sortie pour le certificat signé.
\end{itemize}

Cette commande génère un fichier \texttt{domain.crt} qui contient le certificat signé.
Ce certificat peut être utilisé pour établir des connexions sécurisées (SSL/TLS) sur un serveur web.

\begin{figure}[ht]
    \centering
    \includegraphics[width=0.8\textwidth]{../img/opensslCertSign.png}
    \caption{Signature d'un certificat avec OpenSSL}
    \label{opensslCertSign}
\end{figure}

\subsection{Partie 12}
Pour vérifier le certificat, on utilise la commande suivante:
\begin{Verbatim}[fontsize=\footnotesize]
    openssl x509 -text -noout -in domain.crt
\end{Verbatim}

Disecton cette commande (voir figure \ref{opensslCertVerif} pour la sortie):
\begin{itemize}
    \item \texttt{openssl x509} : Indique que nous voulons travailler avec des certificats X.509.
    \item \texttt{-text} : Affiche les informations du certificat de manière lisible.
    \item \texttt{-noout} : N'affiche pas le contenu brut du certificat.
    \item \texttt{-in domain.crt} : Spécifie le fichier d'entrée (certificat) à vérifier.
\end{itemize}

Cette commande affiche les informations du certificat,
y compris le nom du sujet, la clé publique, la période de validité et d'autres détails.

\begin{figure}[ht]
    \centering
    \includegraphics[width=0.8\textwidth]{../img/opensslCertVerif.png}
    \caption{Vérification d'un certificat avec OpenSSL}
    \label{opensslCertVerif}
\end{figure}


% Conclusion
\section{Observation, interprétation et conclusion}
 [Vos observations et conclusions]
\begin{itemize}
    \item Objectifs atteints ou non
    \item Ce que vous avez accompli
    \item Ce que vous avez compris
\end{itemize}

\end{document}
